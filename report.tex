\documentclass[12pt,letterpaper]{article}
\usepackage{fullpage}
\usepackage[top=2cm, bottom=4.5cm, left=2.5cm, right=2.5cm]{geometry}
\usepackage{amsmath,amsthm,amsfonts,amssymb,amscd}
\usepackage{lastpage}
\usepackage{enumerate}
\usepackage{fancyhdr}
\usepackage{mathrsfs}
\usepackage{xcolor}
\usepackage{graphicx}
\usepackage{listings}
\usepackage{hyperref}
\usepackage{minted}

\hypersetup{%
  colorlinks=true,
  linkcolor=blue,
  linkbordercolor={0 0 1}
}
 
\renewcommand\lstlistingname{High Performance Computing}
\renewcommand\lstlistlistingname{High Performance Computing}
\def\lstlistingautorefname{HPC}

\lstdefinestyle{Python}{
    language        = Python,
    frame           = lines, 
    basicstyle      = \footnotesize,
    keywordstyle    = \color{blue},
    stringstyle     = \color{green},
    commentstyle    = \color{red}\ttfamily
}

\setlength{\parindent}{0.0in}
\setlength{\parskip}{0.05in}

% Edit these as appropriate
\newcommand\course{High Performance Computing}
\newcommand\hwnumber{1}                  % <-- homework number
\newcommand\NetIDa{amr1215}           % <-- NetID of person #1
\newcommand\NetIDb{Agnitha Mohan Ram}           % <-- NetID of person #2 (Comment this line out for problem sets)

\pagestyle{fancyplain}
\headheight 35pt
\lhead{\NetIDa}
\lhead{\NetIDa\\\NetIDb}                 % <-- Comment this line out for problem sets (make sure you are person #1)
\chead{\textbf{Assignment 3}}
\rhead{\course \\ \today}
\lfoot{}
\cfoot{}
\rfoot{\small\thepage}
\headsep 1.5em

\begin{document}

	


\underline{\textbf{1.Matrix-vector operations on a GPU.}} \\\\
I was unable to run in on cuda2 and cuda4. 
Cuda2 was working  initially but when I ran it again, it gave me a segmentation fault for all sizes of matrices and all programs. I'm not sure why that's happening. It was working fine before. \\
Cuda4 was extremely slow in compiling and running. \\\\
\underline{Vector-vector multiplication} \\\\
N = 102400000 \\\\
\begin{tabular}{ |c|c|c|c|c|c| } 
 \hline
 Hostname &Time-CPU&Time-GPU&CPU Bandwidth&GPU Bandwidth&Error \\
 \hline\hline
  cuda1.cims.nyu.edu & 0.613801 s & 0.269058 s & 4.003903 GB/s &12.178802 GB/s & 0 \\
 \hline
cuda2.cims.nyu.edu &  &  &  & & 0 \\
\hline
cuda3.cims.nyu.edu & 2.055223 s & 0.284932 s & 1.195783 GB/s & 11.500284 GB/s & 0 \\
\hline
cuda4.cims.nyu.edu &   &   &  &  & 0 \\
\hline
cuda5.cims.nyu.edu & 0.683609 s  & 0.268576 s & 3.595040 GB/s & 12.200661 GB/s  & 0 \\
 \hline
\end{tabular} \\\\\\
\underline{Vector-Matrix Muliplication}  \\\\
Matrix dimensions: 1000 * 102400\\
Vector dimensions: 102400\\\\

\begin{tabular}{ |c|c|c|c|c|c| } 
 \hline
 Hostname &Time-CPU&Time-GPU&CPU Bandwidth&GPU Bandwidth&Error \\
 \hline\hline
  cuda1.cims.nyu.edu &5.986337  s &  5.986337  s & 4.105349 GB/s &5.895165 GB/s & 0 \\
 \hline
cuda2.cims.nyu.edu &  &  &  & & 0 \\
\hline
cuda3.cims.nyu.edu &13.614585  s & 15.311807  s & 1.805123 GB/s & 2.140048 GB/s& 0 \\
\hline
cuda4.cims.nyu.edu &   &   &  &  & 0 \\
\hline
cuda5.cims.nyu.edu & 4.380782 s   & 5.258230 s& 5.609958 GB/s & 6.231754 GB/s  & 0 \\
 \hline
\end{tabular} \\\\\\

\underline{\textbf{2. 2D Jacobi method on a GPU}} \\\\


\textbf{This was run on cuda3.cims.nyu.edu. Might not work the same way becuase of imcompatiable compute capabilites. }\\


\begin{tabular}{ |c|c|c|c| } 
 \hline
N &Time taken on CPU&Time taken on GPU&Error \\
 \hline\hline
 1000 & 63.530870 & 2.847677 & 0  \\
 \hline
500 & 14.642256 &0.708295 & 0  \\
\hline
100 & 0.341087 &  0.082907  & 0  \\
\hline

\end{tabular} \\\\\\

Norm calculation for max iterations = 1000 \\\\
\begin{tabular}{ |c|c|c|c| } 
 \hline
N &Initial Norm & Final Norm  \\
 \hline\hline
 1000 & 998.0000000 &  948.527721  \\
 \hline
500 & 498.000000 &  448.527534 \\
\hline
100 & 98.000000 &  48.585179  \\
\hline

\end{tabular} \\\\\\
\underline{\textbf{3. Update on final projection}} \\\\
\underline{Completed}
1. Serial Implementation of the techniques \\
Completed the serial implementation of the following:\\
(a) Algorithm by breadth-first search\\ (b) Algorithm by adjacency matrix\\\\
2. OpenMP Implementation of the techniques \\
Completed the OpenMP implementations of the mentioned algorithm above. \\
Achieved a speedup for larger graphs. \\\\
\underline{Next step:}\\
1. MPI version for both algorithm and begin Cuda implementation(April 20th - 26th)\\





\end{document}



